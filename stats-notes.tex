\documentclass{article}
\usepackage[utf8]{inputenc}
\usepackage{amsmath}
\usepackage{enumitem}
\title{Probabilities and Statistics}
\author{hbp253}
\date{January 2020}

\begin{document}

\maketitle

\section{Likelihood vs Probability Density Functions}
    Probability density is how likely a set of points are likely to occur given parameter x
    \newline
    Likelihood takes a data set and represents the likeliness of different parameters for your distribution

\section{Cohen's d}
    You can compare the difference between two groups to the variability within groups with Cohen's d:
    \newline
    \begin{center} 
    $d = \dfrac{\overline{x_1} - \overline{x_2}}{S}$
    \end{center}

    S in this case would be a "pooled standard deviation". Which is a weighed variance between the means:
    \begin{center}
    $S = \sqrt{\dfrac{n_1 * var_1 + n_2 * var_2}{n_1 + n_2}}$
    \end{center}

\section{PMF}
Probability Mass Function or PMF, represent the probability for each unique value. It is normalized by n (the size of the set)
\subsection{The Class Size Paradox}
    Let there be x number of classes for each range of class sizes,\newline
    If each one of the student in the class is asked to count the class size,\newline
    Then the PMF is multiplied by x again before it is normalized.\newline
    Calculating the mean given PMF:
    \begin{center}
    $ \overline{x} = \sum_{i}{} p_i x_i $,
    \end{center}

\section{CDF}


\end{document}